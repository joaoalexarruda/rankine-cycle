\documentclass[
	% -- opções da classe memoir --
	article,			% indica que é um artigo acadêmico
	11pt,				% tamanho da fonte
	oneside,			% para impressão apenas no recto. Oposto a twoside
	a4paper,			% tamanho do papel.
	% -- opções da classe abntex2 --
	%chapter=TITLE,		% títulos de capítulos convertidos em letras maiúsculas
	%section=TITLE,		% títulos de seções convertidos em letras maiúsculas
	%subsection=TITLE,	% títulos de subseções convertidos em letras maiúsculas
	%subsubsection=TITLE % títulos de subsubseções convertidos em letras maiúsculas
	% -- opções do pacote babel --
	english,			% idioma adicional para hifenização
	brazil,				% o último idioma é o principal do documento
	sumario=tradicional
	]{abntex2}


% ---
% PACOTES
% ---

% ---
% Pacotes fundamentais
% ---
\usepackage{lmodern}			% Usa a fonte Latin Modern
\usepackage[T1]{fontenc}		% Selecao de codigos de fonte.
\usepackage[utf8]{inputenc}		% Codificacao do documento (conversão automática dos acentos)
\usepackage{indentfirst}		% Indenta o primeiro parágrafo de cada seção.
\usepackage{nomencl} 			% Lista de simbolos
\usepackage{color}				% Controle das cores
\usepackage{graphicx}			% Inclusão de gráficos
\usepackage{microtype} 			% para melhorias de justificação
% ---
\usepackage{hyperref}
\usepackage{listings}
\usepackage{xcolor}
\usepackage{tcolorbox}
\usepackage{siunitx}
\usepackage{pdfpages}
\usepackage{amsmath}   % Para símbolos matemáticos
\usepackage{booktabs}  % Para melhores linhas de tabela
% ---
% Pacotes adicionais, usados apenas no âmbito do Modelo Canônico do abnteX2
% ---
\usepackage{lipsum}				% para geração de dummy text
% ---

% ---
% Pacotes de citações
% ---
\usepackage[brazilian,hyperpageref]{backref}	 % Paginas com as citações na bibl
\usepackage[alf, abnt-emphasize=bf]{abntex2cite} % Estilo ABNT alfabético% ---

% ---
% Configurações do pacote backref
% Usado sem a opção hyperpageref de backref
\renewcommand{\backrefpagesname}{Citado na(s) página(s):~}
% Texto padrão antes do número das páginas
\renewcommand{\backref}{}
% Define os textos da citação
\renewcommand*{\backrefalt}[4]{
	\ifcase #1 %
		Nenhuma citação no texto.%
	\or
		Citado na página #2.%
	\else
		Citado #1 vezes nas páginas #2.%
	\fi}%
% ---

% --- Informações de dados para CAPA e FOLHA DE ROSTO ---
\titulo{ESTUDO E RESOLUÇÃO DO CICLO DE RANKINE COM MODIFICAÇÕES}
\tituloestrangeiro{}

\autor{
João Alex Arruda da Silva
\\[0.5cm]
Hanna Rodrigues Ferreira}

\local{Brasil}
\data{Fevereiro, 2025}
% ---

% ---
% Configurações de aparência do PDF final

% alterando o aspecto da cor azul
\definecolor{blue}{RGB}{41,5,195}

% informações do PDF
\makeatletter
\hypersetup{
     	%pagebackref=true,
		pdftitle={\@title},
		pdfauthor={\@author},
    	pdfsubject={Modelo de artigo científico com abnTeX2},
	    pdfcreator={LaTeX with abnTeX2},
		pdfkeywords={abnt}{latex}{abntex}{abntex2}{atigo científico},
		colorlinks=true,       		% false: boxed links; true: colored links
    	linkcolor=blue,          	% color of internal links
    	citecolor=blue,        		% color of links to bibliography
    	filecolor=magenta,      		% color of file links
		urlcolor=blue,
		bookmarksdepth=4
}
\makeatother
% ---

% ---
% compila o indice
% ---
\makeindex
% ---

% ---
% Altera as margens padrões
% ---
\setlrmarginsandblock{3cm}{3cm}{*}
\setulmarginsandblock{3cm}{3cm}{*}
\checkandfixthelayout
% ---

% ---
% Espaçamentos entre linhas e parágrafos
% ---

% O tamanho do parágrafo é dado por:
\setlength{\parindent}{1.3cm}

% Controle do espaçamento entre um parágrafo e outro:
\setlength{\parskip}{0.2cm}  % tente também \onelineskip

% Espaçamento simples
\SingleSpacing


% ----
% Início do documento
% ----
\begin{document}

% Seleciona o idioma do documento (conforme pacotes do babel)
%\selectlanguage{english}
\selectlanguage{brazil}

% Retira espaço extra obsoleto entre as frases.
\frenchspacing

% ----------------------------------------------------------
% ELEMENTOS PRÉ-TEXTUAIS
% ----------------------------------------------------------

%---
%
% Se desejar escrever o artigo em duas colunas, descomente a linha abaixo
% e a linha com o texto ``FIM DE ARTIGO EM DUAS COLUNAS''.
% \twocolumn[    		% INICIO DE ARTIGO EM DUAS COLUNAS
%
%---

% página de titulo principal (obrigatório)
\maketitle


% titulo em outro idioma (opcional)



% resumo em português
\begin{resumoumacoluna}
	Este trabalho analisa o ciclo de Rankine, muito utilizado na geração de energia
	nas usinas termelétricas que operam com vapor. São abordados seus processos
	básicos e modificações que aumentam a eficiência térmica, como superaquecimento,
	reaquecimento, regeneração e entre outras. A metodologia inclui uma revisão teórica
	e um estudo de caso em que realizamos cálculos de eficiência térmica, trabalho das
	turbinas e bombas, vazão mássica e construção dos diagramas T-s. Além disso, é
	realizado uma análise paramétrica do desempenho do ciclo. Os resultados nos
	mostram que tais modificações melhoram a eficiência, mas que é necessário um estudo aprofundado.

 \vspace{\onelineskip}

 \noindent
 \textbf{Palavras-chave}: Ciclo de Rankine. Eficiência térmica. Superaquecimento. Diagrama T-s.
\end{resumoumacoluna}

% ----------------------------------------------------------
% ELEMENTOS TEXTUAIS
% ----------------------------------------------------------
\textual

% ----------------------------------------------------------
% Introdução
% ----------------------------------------------------------
\section{Introdução}

Segundo \cite{moran-2018}, o ciclo de Rankine é a estrutura fundamental das usinas termelétricas que operam com vapor. Este é um dos principais ciclos termodinâmicos utilizados na engenharia mecânica para conversão de calor em trabalho, sendo a base para o funcionamento de usinas termoelétricas e outras instalações de geração de energia. Esse ciclo opera com um fluido de trabalho, geralmente água, que passa por processos de aquecimento, expansão, resfriamento e compressão.

Para melhorar a eficiência do Ciclo de Rankine, diversas modificações são adotadas, como o superaquecimento, o reaquecimento e o uso de ciclos supercríticos. Essas modificações têm o objetivo de aumentar a eficiência térmica e reduzir perdas energéticas, tornando as plantas de geração mais sustentáveis e econômicas.

Este trabalho pretende analisar detalhadamente o ciclo de Rankine e suas variações, visando aprofundar o conhecimento sobre sistemas térmicos voltados à produção de energia. Para isso, será realizada uma revisão teórica robusta dos princípios termodinâmicos envolvidos, seguida da aplicação desses conceitos em um estudo de caso prático, que abordará técnicas como reaquecimento, expansão em dois estágios e regeneração térmica.

\section{Revisão bibliográfica}

\subsection{Definição do ciclo de Rankine}

O Ciclo de Rankine é um ciclo termodinâmico idealizado que descreve o funcionamento de uma usina termelétrica convencional. Esse ciclo é composto por quatro processos termodinâmicos: compressão, aquecimento, expansão e resfriamento. A Figura \ref{fig:esquema-simplificado-ciclo-rankine} ilustra o diagrama de um ciclo de Rankine básico.

\begin{figure}[h]
	\centering
	\caption{Esquema simplificado e o diagrama T-S do ciclo Rankine}
	\includegraphics[width=0.7\textwidth]{./images/Esquema simplificado e o diagrama T-S do ciclo Rankine.png}
	\label{fig:esquema-simplificado-ciclo-rankine}
	\fonte{Adaptado de \cite{borgnakke-2020}}
\end{figure}

O fluido de trabalho fica sujeito à seguinte sequência de processos reversíveis internamente: \cite{borgnakke-2020}

\begin{itemize}
	\item \textbf{Processo 1-2}: Processo de bombeamento adiabático reversível na bomba.
	\item \textbf{Processo 2-3}: Transferência de calor a pressão constante na caldeira.
	\item \textbf{Processo 3-4}: Expansão adiabática reversível na turbina (ou em outra máquina motora, tal como a máquina a vapor).
	\item \textbf{Processo 4-1}: Transferência de calor a pressão constante no condensador.
\end{itemize}

Ao desconsiderar as variações de energia cinética e potencial, as trocas de calor e o trabalho líquido do sistema podem ser visualizados como áreas específicas no diagrama temperatura-entropia (T-s). O calor absorvido pelo fluido de trabalho corresponde à área delimitada pelos pontos a-2-2'-3-b-a, enquanto o calor rejeitado pelo fluido é representado pela área a-1-4-b-a. Aplicando a primeira lei da termodinâmica, conclui-se que o trabalho líquido é equivalente à diferença entre essas duas áreas, ou seja, corresponde à região 1-2-2'-3-4-1 no diagrama \cite{borgnakke-2020}.

Na análise do ciclo Rankine, é útil considerar que o rendimento depende da temperatura média na qual o calor é fornecido e da temperatura média na qual o calor é rejeitado. Qualquer variação que aumente a temperatura média na qual o calor é fornecido, ou que diminua a temperatura média na qual o calor é rejeitado, aumentará o rendimento do ciclo Rankine.

\subsection{Componentes Básicos}

Independentemente de um modelo detalhado ou simplificado de usina a vapor baseada no ciclo Rankine, os fundamentos termodinâmicos (conservação de massa/energia, segunda lei e dados termodinâmicos) aplicam-se tanto aos componentes individuais (turbinas, bombas, trocadores de calor) quanto ao ciclo global.

Focando no subsistema mostrado na Figura \ref{fig:planta-exemplo}, modelam-se os quatro componentes principais: turbina, condensador, bomba e caldeira, com água como fluido de trabalho. Usinas a combustíveis fósseis são analisadas como referência, mas os princípios valem para outros tipos.

\begin{figure}[h]
	\centering
	\caption{Planta de potência a vapor acionada por combustível fóssil}
	\includegraphics[width=1.0\textwidth]{./images/planta-exemplo.png}
	\label{fig:planta-exemplo}
	\fonte{\
	cite{moran-2018}}
\end{figure}

No diagrama da Fig. 2, trabalho e calor são positivos conforme as setas, para simplificar a análise usamos algumas hipóteses frequentes, conforme descrito abaixo:
\begin{itemize}
	\item R. P. em todos os componentes.
	\item Energia potencial e cinética desprezível.
    \item Perdas de pressão na caldeira e no condensador desprezíveis.
    \item Bombas e turbinas são consideradas isentrópicas.
\end{itemize}

Será mostrado a seguir a modelagem do ciclo para cada componente do ciclo Rankine, conforme é exibido por \citeonline{moran-2018}

\subsubsection{Turbina}

A turbina é o componente que converte a energia térmica do vapor em trabalho mecânico. O vapor entra na turbina com uma pressão e temperatura elevadas e sai com pressão e temperatura menores. O trabalho líquido da turbina é a diferença entre o trabalho de entrada e saída, conforme a equação \ref{eq:trabalho-turbina}.

\begin{equation}
	\dot{W}_{\text{turbina}} = \dot{m}(h_1 - h_2)
	\label{eq:trabalho-turbina}
\end{equation}

\subsubsection{Condensador}

O condensador é o componente que converte o vapor em água líquida, rejeitando calor para o ambiente. O calor rejeitado pelo condensador é a diferença entre o calor de entrada e saída, conforme a equação \ref{eq:calor-condensador}.

\begin{equation}
	\dot{Q}_{\text{condensador}} = \dot{m}(h_2 - h_3)
	\label{eq:calor-condensador}
\end{equation}

\subsubsection{Bomba}

A bomba é o componente que comprime a água líquida, aumentando sua pressão. O trabalho líquido da bomba é a diferença entre o trabalho de entrada e saída, conforme a equação \ref{eq:trabalho-bomba}.

\begin{equation}
	\dot{W}_{\text{bomba}} = \dot{m}(h_4 - h_3)
	\label{eq:trabalho-bomba}
\end{equation}

\subsubsection{Caldeira}

A caldeira é o componente que converte a água líquida em vapor, absorvendo calor do ambiente. O calor absorvido pela caldeira é a diferença entre o calor de entrada e saída, conforme a equação \ref{eq:calor-caldeira}.

\begin{equation}
	\dot{Q}_{\text{caldeira}} = \dot{m}(h_1 - h_2)
	\label{eq:calor-caldeira}
\end{equation}

\subsection{Parâmetros de Desempenho}

\subsubsection{Eficiência Térmica}

A eficiência térmica do ciclo Rankine é dada pela razão entre o trabalho líquido produzido e o calor fornecido na caldeira, conforme a equação \ref{eq:eficiencia-termica}.

\begin{equation}
	\eta = \frac{\dot{W}_{\text{turbina}}}{\dot{Q}_{\text{caldeira}}}
	\label{eq:eficiencia-termica}
\end{equation}

\subsubsection{Taxa de Calor}

A taxa de calor representa a quantidade de energia térmica fornecida ao sistema (geralmente medida em Btu) necessária para gerar uma unidade de trabalho útil produzido pelo ciclo (normalmente expresso em kWh). Por isso, ela é definida como a razão entre a energia térmica consumida e o trabalho líquido gerado, com unidades de Btu/kWh. Essa taxa tem uma relação inversa com a eficiência termodinâmica do ciclo: quanto maior a eficiência, menor a quantidade de calor requerida para produzir a mesma quantidade de trabalho.

\subsubsection{Back work ratio}

\begin{figure}[h]
	\centering
	\caption{Planta de potência a vapor acionada por combustível fóssil}
	\includegraphics[width=0.5\textwidth]{./images/trabalho-realizado.png}
	\label{fig:trabalho-realizado}
	\fonte{\cite{moran-2018}}
\end{figure}

O back work ratio (bwr) é um parâmetro que quantifica a relação entre o trabalho consumido pela bomba e o trabalho gerado pela turbina no ciclo de potência. Para o sistema da Figura \ref{fig:trabalho-realizado}, usando as equações já definidas para o trabalho da bomba e da turbina, o bwr é calculado pela equação \ref{eq:bwr}.

\begin{equation}
	\text{bwr} = \frac{\dot{W}_{\text{bomba}}}{\dot{W}_{\text{turbina}}}
	\label{eq:bwr}
\end{equation}

\subsection{Aplicações do Ciclo de Rankine}

Dentre os sete tipos de usinas de energia que operam com base em ciclos termodinâmicos, seis estão diretamente vinculadas ao ciclo de Rankine. Esse ciclo é o elemento fundamental das usinas termelétricas a vapor, servindo como modelo central para conversão de calor em trabalho mecânico ou elétrico. Sua versatilidade permite aplicações em sistemas que vão desde usinas nucleares e movidas a combustíveis fósseis até fontes renováveis, como geotérmica e solar térmica, consolidando-o como pilar da geração de energia em larga escala.

A Figura \ref{fig:etn} mostra a turbina de Angra 2, uma usina nuclear que opera com base no ciclo de Rankine. A usina é composta por um reator nuclear, que gera calor, e uma turbina a vapor, que converte esse calor em energia elétrica. O ciclo de Rankine é responsável por transferir o calor do reator para a turbina, garantindo a eficiência do processo.

\begin{figure}[h]
	\centering
	\caption{Turbina de Angra 2}
	\includegraphics[width=0.7\textwidth]{./images/etn.jpg}
	\label{fig:etn}
	\fonte{\cite{Eletronuclear_Imagem_Angra}}
\end{figure}

\section{Parâmetros de influência na eficiência do ciclo de Rankine}

\subsection{Efeitos Da Pressão E Da Temperatura No Ciclo Rankine}

No ciclo Rankine, a pressão e a temperatura afetam diretamente o rendimento e o trabalho líquido realizado no ciclo. Em seguida, examinam-se os principais impactos dessas variáveis, juntamente com suas respectivas representações gráficas.

\subsection{Efeito da Pressão na Saída da Turbina}

No diagrama T-s da Figura \ref{fig:efeito-pressao}, observa-se o efeito da redução da pressão na saída da turbina, de P4 para P4'. Essa redução causa uma diminuição na temperatura na qual o calor é rejeitado. O aumento do trabalho líquido é representado pela área 1-4-4'-1'-2-2'-1, enquanto o aumento do calor transferido ao fluido corresponde à área a'-2'-2-a. Como essas duas áreas são aproximadamente iguais, o rendimento do ciclo aumenta, devido à redução da temperatura média de rejeição de calor. A redução da pressão também diminui o título do vapor na saída da turbina. Se a umidade ultrapassar 10\%, pode ocorrer erosão das palhetas e queda na eficiência \cite{borgnakke-2020}.

\begin{figure}[h]
	\centering
	\caption{Efeito da pressão de descarga da turbina sobre o rendimento do ciclo Rankine}
	\includegraphics[width=0.4\textwidth]{./images/efeito-pressao.png}
	\label{fig:efeito-pressao}
	\fonte{\cite{borgnakke-2020}}
\end{figure}

\subsubsection{Efeito do Superaquecimento do Vapor}

Na Figura \ref{fig:efeito-vapor}, é mostrado o efeito do superaquecimento do vapor. O trabalho líquido aumenta correspondendo à área 3-3'-4'-4-3, e o calor transferido na caldeira aumenta com a área 3-3'-b'-b-3. Como a relação entre essas áreas é maior que a relação entre o trabalho líquido e o calor fornecido no restante do ciclo, o superaquecimento do vapor resulta em um aumento do rendimento do ciclo Rankine. O superaquecimento aumenta a temperatura média de transferência de calor e melhora o título do vapor na saída da turbina \cite{borgnakke-2020}.

\begin{figure}[h]
	\centering
	\includegraphics[width=0.4\textwidth]{./images/efeito-vapor.png}
	\caption{Efeito do superaquecimento do vapor sobre o rendimento do ciclo Rankine}
	\label{fig:efeito-vapor}
	\fonte{\cite{borgnakke-2020}}
\end{figure}

\subsubsection{Efeito da Pressão Máxima do Vapor}

A Figura \ref{fig:efeito-max-vapor} ilustra o impacto do aumento da pressão máxima do vapor. Mantendo-se constantes a temperatura máxima e a pressão de saída da turbina, o calor rejeitado diminui com a área b'-4'-4-b-b'. O trabalho líquido aumenta com a área hachurada simples e diminui com a área duplamente hachurada. O rendimento do ciclo aumenta com o aumento da pressão, pois o calor rejeitado diminui e a temperatura média de fornecimento de calor aumenta

\begin{figure}[h]
	\centering
	\includegraphics[width=0.4\textwidth]{./images/efeito-max-vapor.png}
	\caption{Efeito da pressão máxima do vapor sobre o rendimento do ciclo Rankine}
	\label{fig:efeito-max-vapor}
	\fonte{\cite{borgnakke-2020}}
\end{figure}

A Figura \ref{fig:efeito-pressao-temperatura} mostra como a pressão e a temperatura afetam o trabalho do ciclo Rankine. Já a Figura \ref{fig:efeito-pressao-temperatura-eficiencia} apresenta a influência dessas variáveis na eficiência do ciclo. Ambas destacam os efeitos combinados das variáveis.

\begin{figure}[h]
	\centering
	\includegraphics[width=0.4\textwidth]{./images/efeito-pressao-temperatura.png}
	\caption{Efeito da pressão e da temperatura no trabalho do ciclo Rankine}
	\label{fig:efeito-pressao-temperatura}
	\fonte{\cite{borgnakke-2020}}
\end{figure}

\begin{figure}
	\centering
	\includegraphics[width=0.4\textwidth]{./images/efeito-pressao-temperatura-eficiencia.png}
	\caption{Efeito da pressão e da temperatura na eficiência do ciclo Rankine}
	\label{fig:efeito-pressao-temperatura-eficiencia}
	\fonte{\cite{borgnakke-2020}}
\end{figure}

Esses três fatores combinados podem otimizar o ciclo Rankine, desde que o projeto evite problemas como erosão das palhetas da turbina devido a altos níveis de umidade.
Além dessas considerações, observamos que o ciclo é representado por quatro processos conhecidos (dois isobáricos e dois isentrópicos) que se desenrolam entre os quatro estados, abrangendo um total de oito características. Assumindo que o estado 1 seja um estado líquido saturado (x1 = 0), precisamos definir três parâmetros (8-4-1). A pressão operacional é controlada fisicamente pela alta pressão produzida pela bomba, P2 = P3, o superaquecimento para T3 (ou x3 = 1, se não houver superaquecimento) e a temperatura do condensador T1, que é o resultado da transferência de calor que acontece \cite{borgnakke-2020}.

\subsection{Interpretação Dos Efeito Das Pressões Da Caldeira E Do Condensador}

A Figura \ref{fig:efeitos-caldeira} exibe dois ciclos ideais submetidos à mesma pressão. Contudo, com pressões distintas na caldeira. Conforme a análise, a temperatura média do calor adicionado é maior no ciclo de pressão mais elevada 1' - 2' - 3' - 4' - 1' do que no ciclo 1 - 2 - 3 - 4 - 1. Portanto, o aumento da pressão na caldeira do ciclo ideal de Rankine tende a eficiência térmica.

\begin{figure}[h]
	\centering
	\includegraphics[width=0.4\textwidth]{./images/efeitos-caldeira.png}
	\caption{Efeitos da variação das pressões de operação do ciclo ideal Rankine na caldeira}
	\label{fig:efeitos-caldeira}
	\fonte{\cite{moran-2018}}
\end{figure}

A Figura \ref{fig:efeito-condensador} ilustra dois ciclos com pressões idênticas na caldeira, mas com duas pressões distintas no condensador. Um condensador funciona sob a pressão atmosférica, enquanto o outro opera sob uma pressão inferior à atmosfera. Para os ciclos 1-2-3-4-1 que condensam sob pressão atmosférica, a temperatura de rejeição de calor é de 100 °C (212 °F). A temperatura do calor devolvido para o ciclo de pressão mais baixa 1 - 2"- 3"- 4"- 1 é menor, resultando em uma maior eficiência térmica para este ciclo. Portanto, conclui-se que a redução da pressão no compressor tende a incrementar a eficiência térmica.

\begin{figure}[h]
	\centering
	\includegraphics[width=0.5\textwidth]{./images/efeito-condensador.png}
	\caption{Efeitos da variação das pressões de operação do ciclo ideal Rankine no condensador}
	\label{fig:efeito-condensador}
	\fonte{\cite{moran-2018}}
\end{figure}

O condensador opera com a pressão de saturação equivalente à temperatura ambiente, pois esta é a temperatura ideal para a dissipação de calor para as proximidades. A finalidade de manter a pressão de exaustão mais baixa possível na turbina é a principal razão para a inclusão do condensador em uma instalação de potência. A caldeira poderia ser abastecida com água líquida à pressão atmosférica por meio da bomba, enquanto o vapor poderia ser liberado diretamente no ar ao sair da turbina. No entanto, ao incorporar um condensador, que opera a vapor a uma pressão inferior à atmosférica, a turbina terá uma área de pressão mais baixa onde será feita a descarga, o que resultará em um aumento do trabalho líquido da eficiência térmica. Incorporando um condensador adicional, o fluido de trabalho funcionará em circuito fechado, garantindo uma circulação constante do fluido de trabalho.

\subsection{Efeito da temperatura na eficiência térmica}

Basicamente o ciclo ideal Rankine consiste em processos onde existem reversibilidades internas, o que nos possibilita obter uma expressão para eficiência térmica em função das temperaturas médias durante o processo de interação térmica. A eficiência térmica do ciclo ideal Rankine é expressa pela equação \ref{eq:eficiencia-termica-temperatura}.

\begin{equation}
	\eta_{ideal} = 1 - \frac{T_{sai}}{T_{ent}}
	\label{eq:eficiencia-termica-temperatura}
\end{equation}

Pode-se concluir que a eficiência térmica do ciclo ideal tende a aumentar quando a temperatura média pela qual a energia é adicionada por transferência de calor aumenta ou a temperatura pela qual a energia rejeitada diminui.

\section{Ciclos de Rankine Modificados}

\subsection{Tipos de modificações}

O ciclo de Rankine pode ser modificado de várias maneiras para melhorar sua eficiência e desempenho. As modificações mais comuns incluem o superaquecimento, o reaquecimento e a regeneração térmica. Cada uma dessas técnicas tem o objetivo de aumentar a eficiência térmica do ciclo, reduzindo as perdas de calor e melhorando a qualidade do vapor na saída da turbina.

\subsubsection{Reaquecimento}

O ciclo Rankine com reaquecimento foi projetado para aproveitar o aumento de rendimento proporcionado por pressões mais altas, evitando umidade excessiva nos estágios de baixa pressão da turbina. Conforme ilustrado na Figura \ref{fig:reaquecimento}, o vapor inicialmente se expande até uma pressão intermediária na turbina, sendo reaquecido na caldeira antes de expandir novamente até a pressão de saída. Embora o diagrama T-s demonstre que o reaquecimento proporciona um pequeno ganho de rendimento devido à pequena variação na temperatura média de fornecimento de calor, sua principal vantagem reside na redução da umidade nos estágios finais da turbina. O autor destaca ainda que, caso os metais permitam um superaquecimento adequado do vapor até 3', o ciclo Rankine simples seria mais eficiente que o ciclo com reaquecimento, tornando este último desnecessário \cite{borgnakke-2020}.

\begin{figure}[h]
	\centering
	\includegraphics[width=0.65\textwidth]{./images/reaquecimento.png}
	\caption{Ciclo Rankine com reaquecimento}
	\label{fig:reaquecimento}
	\fonte{\cite{borgnakke-2020}}
\end{figure}

\subsubsection{Regeneração}

O ciclo Rankine regenerativo é uma importante variação que utiliza aquecedores da água de alimentação para melhorar a eficiência do sistema. Como mostrado na Figura \ref{fig:regerenacao}, no ciclo sem superaquecimento, o fluido de trabalho é aquecido na fase líquida entre os estados 2 e 2', com uma temperatura média significativamente menor em comparação ao processo de vaporização (2'-3). Isso resulta em uma temperatura média de transferência de calor inferior à do ciclo de Carnot (1'-2'-3-4-1), acarretando um rendimento menor. No ciclo regenerativo, o fluido entra na caldeira em um estado intermediário entre 2 e 2', aumentando a temperatura média de fornecimento de calor e, consequentemente, o rendimento do ciclo \cite{borgnakke-2020}.

\begin{figure}[h]
	\centering
	\includegraphics[width=0.5\textwidth]{./images/regeneracao.png}
	\caption{Diagrama T-s que mostra a relação entre os rendimentos dos ciclos de Carnot e Rankine.}
	\label{fig:regerenacao}
	\fonte{\cite{borgnakke-2020}}
\end{figure}

O ciclo regenerativo ideal, como ilustrado na Figura \ref{fig:regenerativo}, apresenta uma característica singular em comparação ao ciclo Rankine. Após a saída da bomba, o líquido circula ao redor da carcaça da turbina em sentido contrário ao do vapor, permitindo a transferência de calor do vapor para o líquido de forma teoricamente reversível. Nesse cenário ideal, a linha 4-5 no diagrama T-s, que representa o escoamento do vapor pela turbina, é paralela à linha 1-2-3, que indica o processo de bombeamento e o escoamento do líquido ao redor da turbina. As áreas 2-3-b-a-2 e 5-4-d-c-5 são congruentes e representam o calor transferido entre vapor e líquido. Esse ciclo apresenta rendimento térmico equivalente ao do ciclo de Carnot, já que a área 1-5-c-a-1 é igual à área de calor rejeitado do ciclo de Carnot \cite{borgnakke-2020}.

\begin{figure}[h]
	\centering
	\includegraphics[width=0.65\textwidth]{./images/regenerativo.png}
	\caption{Ciclo Rankine regenerativo ideal}
	\label{fig:regenerativo}
	\fonte{\cite{borgnakke-2020}}
\end{figure}

Na prática, no entanto, a implementação desse ciclo ideal é inviável devido à dificuldade de realizar uma transferência de calor eficiente na turbina e ao aumento significativo da umidade do vapor na saída. O ciclo regenerativo real, mostrado na Figura \ref{fig:reg-aquecedor}, resolve essa limitação com a extração de parte do vapor parcialmente expandido na turbina, que é direcionado a aquecedores da água de alimentação. O líquido condensado é bombeado para se misturar ao vapor extraído, resultando em uma mistura saturada no estado 3. Para atingir a pressão da caldeira, uma segunda bomba é necessária. A vantagem principal desse ciclo é o aumento da temperatura média na qual o calor é fornecido ao fluido de trabalho, melhorando a eficiência térmica em comparação ao ciclo Rankine convencional \cite{borgnakke-2020}.

\begin{figure}[h]
	\centering
	\includegraphics[width=0.65\textwidth]{./images/regenerativo-aquecedor.png}
	\caption{Ciclo regenerativo com aquecedor de água de alimentação de mistura}
	\label{fig:reg-aquecedor}
	\fonte{\cite{borgnakke-2020}}
\end{figure}

\subsubsection{Cogeração}

A cogeração é uma aplicação industrial do ciclo de potência a vapor que combina a geração de eletricidade com o suprimento de energia térmica para processos produtivos. Como mostrado na Figura \ref{fig:cogeracao}, o vapor expandido até uma pressão intermediária na turbina de alta pressão é utilizado como fonte de energia para o processo produtivo, eliminando a necessidade de uma segunda caldeira dedicada. Esse vapor pode atender demandas específicas, como aquecer ambientes ou fornecer vapor para processos industriais \cite{borgnakke-2020}.

\begin{figure}[h]
	\centering
	\includegraphics[width=0.65\textwidth]{./images/cogeracao.png}
	\caption{Ciclo Rankine com cogeração}
	\label{fig:cogeracao}
	\fonte{\cite{borgnakke-2020}}
\end{figure}

A cogeração pode operar em diferentes arranjos: em algumas instalações, o vapor é o produto principal e a eletricidade é um subproduto, típico de fábricas e pequenas unidades; em outros casos, como em empresas de geração elétrica, a eletricidade é o foco principal, com o vapor como subproduto. Exemplos incluem plantas que fornecem eletricidade para a rede e água quente para aquecimento domiciliar, desde que a densidade populacional e as distâncias de distribuição sejam favoráveis.
A cogeração de energia pode ser amplamente aplicada na indústria alimentícia, oferecendo múltiplos benefícios por meio da integração de processos. Por exemplo, sistemas de cogeração permitem gerar eletricidade para alimentar equipamentos, reduzindo a dependência da rede externa e garantindo estabilidade energética, enquanto subprodutos como vapor são aproveitados em etapas de cozimento, esterilização ou secagem de alimentos. Além disso, a água quente produzida pelo sistema atende a demandas de limpeza e sanitização, e a água gelada, gerada por chillers de absorção a partir de 5°C, é empregada no resfriamento de produtos e climatização. O ar quente derivado do processo auxilia em etapas como secagem, e até mesmo a produção de $CO_2$ de alta pureza para bebidas é viabilizada com custos reduzidos. Dessa forma, a cogeração otimiza recursos, diminui despesas com energia e aumenta a eficiência operacional, integrando necessidades térmicas, elétricas e industriais em um único sistema sustentável \cite{Ecogen_Cogeracao_2025}.

\subsubsection{Superaquecimento}

No processo de superaquecimento não existe limitação quanto à presença de vapor saturado na entrada da turbina, possibilitando adicionar mais energia ao vapor através da transferência de calor, fazendo com que a condição de vapor superaquecido eleve antes de entrar na turbina. O equipamento chamado de superaquecedor que gera essa energia extra que trabalha em conjunto com a caldeira, formando o gerador de vapor. Na Figura \ref{fig:superaquecido}, é mostrado um ciclo ideal de Rankine com vapor superaquecido na entrada da turbina, representado pelo ciclo 1'-2'-3-4-1' que em comparação ao ciclo sem superaquecimento (ciclo 1-2-3-4-1) possui uma temperatura média mais elevada durante a adição de calor, o que aumenta a eficiência térmica do sistema.
Outra vantagem é que o título do vapor no estado 2' (saída da turbina) é maior do que no estado 2 (sem superaquecimento). Isso reduz o problema associado ao baixo título do vapor na saída da turbina, comum em ciclos sem superaquecimento. Com isso, um controle adequado do superaquecimento, é possível até mesmo garantir que o vapor na saída da turbina permaneça na região de vapor superaquecido.

\begin{figure}[h]
	\centering
	\includegraphics[width=0.5\textwidth]{./images/superaquecido.png}
	\caption{Ciclo Rankine com superaquecimento}
	\label{fig:superaquecido}
	\fonte{\cite{moran-2018}}
\end{figure}


\subsubsection{Ciclo Supercrítico}

Segundo \citeonline{moran-2018}, a temperatura e a pressão do vapor em turbinas são limitadas pelas propriedades dos materiais empregados em componentes como superaquecedores, reaquecedores e nas próprias turbinas. Pressões elevadas demandam tubulações capazes de suportar altas tensões e temperaturas. Avanços em materiais e métodos de fabricação permitiram elevar esses limites, aumentando a eficiência térmica dos ciclos de geração de energia, reduzindo o consumo de combustível e os impactos ambientais. Conforme ilustrado na Figura \ref{fig:supercritico}, que representa um ciclo ideal de reaquecimento, em plantas supercríticas, a geração de vapor ocorre acima da pressão crítica da água (22,1 MPa), sem transição brusca entre líquido e vapor, como destacado no Processo 6-1. Nesses sistemas, a água é aquecida gradualmente em tubulações, sem o processo convencional de ebulição, utilizando carvão pulverizado como fonte de energia.

\begin{figure}[h]
	\centering
	\includegraphics[width=0.5\textwidth]{./images/supercritico.png}
	\caption{Ciclo Rankine supercrítico}
	\label{fig:supercritico}
	\fonte{\cite{moran-2018}}
\end{figure}

Plantas supercríticas modernas operam com pressões próximas a 30 MPa e temperaturas de 600 °C, alcançando eficiências térmicas de até 47\%. Com o desenvolvimento de superligas resistentes à corrosão e altas temperaturas, plantas ultrassupercríticas atingem 35 MPa e 750 °C, superando 50\% de eficiência. Em contraste, usinas subcríticas têm eficiência máxima de cerca de 40\%. Embora o custo inicial das plantas supercríticas seja maior, o menor consumo de combustível compensa economicamente a longo prazo. Além disso, a redução no uso de combustível diminui emissões de $CO_2$, outros poluentes e resíduos sólidos, tornando essas tecnologias mais sustentáveis. Esse avanço, exemplificado pelas figuras mencionadas, demonstra como inovações técnicas promovem eficiência energética, economia de recursos e menor impacto ambiental \cite{moran-2018}.

\subsection{Desempenho dos ciclos modificados}

Os ciclos de potência a vapor desempenham um papel fundamental na conversão de energia térmica em trabalho útil. O ciclo de Rankine simples é amplamente utilizado, mas sua eficiência pode ser melhorada por modificações. A seguir três formas de realizar isso para o ciclo de Rankine simples ideal \cite{cengel-2008}.

\subsubsection{Comparação da Eficiência dos Ciclos Modificados com o Ciclo Simples}

A eficiência térmica do ciclo de Rankine pode ser aumentada por três principais modificações:

\subsubsubsection{Redução da pressão no condensador}

Como ilustrado na Figura \ref{fig:reducao-pressao}, o efeito da diminuição da pressão no condensador sobre a eficiência do ciclo de Rankine, o estado de entrada na turbina é mantido o mesmo. A área colorida desse diagrama representa o aumento do trabalho líquido devido à diminuição da pressão do condensador de P4 para P4'. O consumo de calor também aumenta (área sob a curva 2'-2), mas o aumento é muito pequeno. Deste modo, o efeito global da diminuição da pressão no condensador é um aumento na eficiência térmica do ciclo.

\begin{figure}[h]
	\centering
	\includegraphics[width=0.3\textwidth]{./images/reducao-pressao.png}
	\caption{Efeito da redução da pressão no condensador sobre a eficiência do ciclo de Rankine}
	\label{fig:reducao-pressao}
	\fonte{\cite{cengel-2008}}
\end{figure}

\subsubsubsection{Superaquecimento do vapor}

O superaquecimento do vapor permite aumentar a temperatura média de adição de calor sem elevar a pressão da caldeira, melhorando a eficiência dos ciclos de potência a vapor. No diagrama T-s, conforme Figura \ref{fig:superaqueci}, isso se reflete no aumento do trabalho líquido e do calor fornecido. Além disso, temperaturas mais altas reduzem a umidade do vapor na saída da turbina, como mostra o maior título no estado 4' em relação ao estado 4.

\begin{figure}[h]
	\centering
	\includegraphics[width=0.3\textwidth]{./images/superaqueci.png}
	\caption{O efeito do superaquecimento do vapor a temperaturas mais altas no ciclo de Rankine ideal}
	\label{fig:superaqueci}
	\fonte{\cite{cengel-2008}}
\end{figure}

\subsubsubsection{Aumento da pressão na caldeira}

O efeito do aumento da pressão da caldeira sobre o desempenho dos ciclos de potência a vapor é visto no diagrama T-s exibido na Figura \ref{fig:pressao-caldeira}, observe que para uma temperatura fixa na entrada da turbina, o ciclo se desloca para a esquerda e o conteúdo de umidade do vapor na saída da turbina aumenta. Esse efeito colateral indesejado pode ser corrigido pelo reaquecimento do vapor.

\begin{figure}[h]
	\centering
	\includegraphics[width=0.3\textwidth]{./images/pressao-caldeira.png}
	\caption{Efeito do aumento da pressão da caldeira sobre o desempenho dos ciclos de potência a vapor}
	\label{fig:pressao-caldeira}
	\fonte{\cite{cengel-2008}}
\end{figure}

Os efeitos da redução da pressão no condensador, do superaquecimento do vapor a temperaturas mais altas e do aumento da pressão da caldeira sobre a eficiência térmica do ciclo Rankine é mostrado no diagrama T-s contido na Figura \ref{fig:diagramas-ts-3-ciclos}, resultado de um exemplo resolvido do livro \cite{cengel-2008}.

\begin{figure}[h]
	\centering
	\includegraphics[width=1.0\textwidth]{./images/diagramas-ts-3-ciclos.png}
	\caption{Diagramas T-s dos três ciclos de potência a vapor}
	\label{fig:diagramas-ts-3-ciclos}
	\fonte{\cite{cengel-2008}}
\end{figure}

O exemplo analisa os efeitos das alterações na pressão da caldeira e na temperatura de superaquecimento sobre a eficiência térmica de uma usina operando no ciclo de Rankine ideal, onde vapor entra na turbina a 3 MPa e 350 °C e é condensado no condensador à pressão de 10 KPa.
Na Figura \ref{fig:diagramas-ts-3-ciclos} (a), a redução da pressão no condensador de 75 kPa para 10 kPa eleva a eficiência térmica de 26\% para 33,4\%. No entanto, essa mudança também aumenta a umidade do vapor na saída da turbina, passando de 11,4\% para 18,7\%, o que pode causar danos às pás da turbina.
Na Figura \ref{fig:diagramas-ts-3-ciclos} (b), considera o superaquecimento do vapor, comparando temperaturas de 350 °C e 600 °C a 3 MPa. Com essa alteração, a eficiência térmica sobe para 37,3\%, além de reduzir a umidade do vapor para 8,5\%, minimizando o risco de erosão nas pás da turbina.
Na Figura \ref{fig:diagramas-ts-3-ciclos} (c), o aumento da pressão da caldeira de 3 MPa para 15 MPa, mantendo a temperatura fixa em 600 °C, faz a eficiência térmica atingir 43\%, um valor próximo ao limite teórico. Entretanto, essa mudança eleva novamente a umidade do vapor na saída da turbina para 19,6\%, tornando necessário o uso de técnicas como reaquecimento para mitigar esse efeito.

\section{Resolução do Problema Proposto}

\subsection{Cálculos}
\label{sec:calculos}

Todos os cálculos foram realizados utilizando a linguagem de programação Python, com o auxílio das bibliotecas NumPy e SciPy. O código-fonte está disponível no repositório do projeto \cite{Arruda_Rankine_2025}.

Além disso, há um documento complementar contendo os cálculos realizados para a resolução do problema proposto. O arquivo foi anexado ao final deste relatório.

\subsection{Eficiência Térmica do Sistema}

$$\eta = 48.63\%$$

\subsection{Taxas de Trabalho}

\begin{table}[ht]
	\centering
	\caption{Resultados dos cálculos termodinâmicos}
	\label{tab:calculos}
	\begin{tabular}{lS[table-format=4.2]}
	\toprule
	{Parâmetro} & {Valor (\si{\kilo\watt})} \\
	\midrule
	$W_{\text{turbina 1}}$               & 1192.38 \\
	$W_{\text{turbina 2,1'}}$ & 1506.24 \\
	$W_{\text{turbina 2,2'}}$ & 1343.52 \\
	$W_{\text{bomba 1}}$                  &    0.97 \\
	$W_{\text{bomba 2}}$                  &   35.30 \\
	$W_{\text{bomba 3}}$                  &    6.85 \\
	$Q_{\text{cald 1}}$                   & 7036.30 \\
	$Q_{\text{cald 2}}$                   & 1186.71 \\
	$Q_{\text{cond}}$                     & 4248.32 \\
	\bottomrule
	\end{tabular}
	\end{table}

\subsection{Vazão Mássica de Água de Resfriamento}

$$\dot{m}_{\text{água}} = 67756.30 \, \si{\kilo\gram\per\second}$$

\subsection{Diagrama T x s}

A Figura \ref{fig:ts} apresenta o diagrama T x s do ciclo de Rankine proposto.

\begin{figure}[h]
	\centering
	\includegraphics[width=0.9\textwidth]{./images/ts.png}
	\caption{Diagrama T x s do ciclo de Rankine proposto}
	\label{fig:ts}
	\fonte{Elaborado pelo autor}
\end{figure}

\subsection{Análise Paramétrica}

A Figura \ref{fig:analise_parametrica} apresenta a análise paramétrica realizada para o ciclo de Rankine proposto. Variou-se a pressão de descarga em todas as turbinas e a temperatura de saída da caldeira 1.

\begin{figure}[h]
	\centering
	\includegraphics[width=0.9\textwidth]{./images/analise_parametrica.png}
	\caption{Análise paramétrica do ciclo de Rankine proposto}
	\label{fig:analise_parametrica}
	\fonte{Elaborado pelo autor}
\end{figure}

\section{Resultados e Discussão}

Os resultados obtidos demonstram uma eficiência térmica de 48,63\% para o ciclo proposto, valor significativamente superior ao ciclo de Rankine simples (tipicamente 30-40\%), evidenciando o impacto positivo das modificações implementadas.

A Tabela \ref{tab:calculos} revela que as turbinas contribuem com 90,7\% do trabalho líquido total $\sum{W_{turb}} = 4.042,14 kW$, enquanto as bombas consomem apenas 1,4\% do trabalho gerado $\sum{W_{bomb}} = 43,12 kW$, resultando em um \textit{back work ratio} de 0,0107. Esta relação favorável caracteriza sistemas de potência a vapor, onde o trabalho de compressão é mínimo comparado à expansão.

A vazão mássica de água de resfriamento de 67.756 kg/s destaca um desafio operacional crítico: mesmo com a regeneração térmica, que reduz a carga térmica no condensador, ainda são necessários volumes colossais de água para rejeição de calor. Em ambientes com restrições hídricas, isto exigiria o uso de torres de resfriamento ou sistemas de recirculação, aumentando a complexidade e custos.

Comparando com dados da literatura, a eficiência alcançada situa-se na faixa superior de plantas supercríticas modernas (45-50\%), validando a configuração proposta. Entretanto, o modelo não considera:

\begin{itemize}
	\item Perdas por atrito em tubulações e turbinas (aproximadamente 5-10\%)
	\item Consumo energético de sistemas auxiliares (aproximadamente 3-5\%)
	\item Degradação de desempenho ao longo do tempo
\end{itemize}

Estes fatores sugerem que a eficiência real operacional seria (35-40\%), ainda competitiva frente a ciclos convencionais. Para aplicações práticas, recomenda-se:
\begin{itemize}
	\item Estudo econômico do custo incremental das modificações
	\item Análise exergética para identificar pontos de maior irreversibilidade
	\item Simulação dinâmica para avaliar resposta a variações de carga
\end{itemize}

% ---
% Finaliza a parte no bookmark do PDF, para que se inicie o bookmark na raiz
% ---
\bookmarksetup{startatroot}%
% ---

% ---
% Conclusão
% ---
\section{Considerações finais}

O ciclo de Rankine é um dos ciclos termodinâmicos mais utilizados em plantas de geração de energia elétrica. A eficiência térmica do ciclo é um dos principais parâmetros de interesse, pois está diretamente relacionada ao consumo de combustível e à emissão de gases poluentes. Neste trabalho, foi proposto um ciclo de Rankine ideal com duas turbinas e três bombas, e foram realizados cálculos termodinâmicos para determinar a eficiência do ciclo, as taxas de trabalho e a vazão mássica de água de resfriamento. Além disso, foi elaborado um diagrama T x s do ciclo e realizada uma análise paramétrica para avaliar o efeito da pressão de descarga das turbinas e da temperatura de saída da caldeira 1 sobre a eficiência do ciclo.


\section{Anexo: Documento Complementar}
\clearpage
\label{anexo:pdf}

\includepdf[pages=-, scale=1.0]{Rankine-Modificado.pdf}

% ----------------------------------------------------------
% ELEMENTOS PÓS-TEXTUAIS
% ----------------------------------------------------------
\postextual

% ----------------------------------------------------------
% Referências bibliográficas
% ----------------------------------------------------------
\clearpage
\bibliography{references}


\end{document}
